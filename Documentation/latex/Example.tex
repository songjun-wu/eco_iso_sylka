\chapter{Case study}



The folder named \textsf{CaseStudy}, which is distributed with the the \echo model package, includes a sample dataset. This dataset is used in this case study to illustrate the process of creating a database and running \echo for a mountain watershed. We will use the GIS functionality provided by PCRaster to assist us in the construction of the database. It is a good idea to have the PCRaster documentation open in a browser tab and peruse it to learn more about the commands we will be using in these examples.

\section{Configuration file} 

The configuration file presents the main communication interface with the model. The configuration file is a plain text file with pairs of keys and values. The values indicate options, paths to folders, or name of files that contain information needed by \echo.   The easiest way to set the configuration file is to generate a template that can subsequently be edited. To generate a configuration file template, navigate to the \textit{CaseStudy} folder and use the following command:

\begin{verbatim}
ech2o -g config.ini
\end{verbatim}


Where the '\textsf{-g}' option indicates that we wish to generate a configuration file with the name \textsf{config.ini}.  In the configuration file we provide information about the location of the database, the names of the files and also control the length of the run, the size of the time step, the outputs we want the model to produce and select a number of options.  

Open the file \textsf{config.ini} with any text editor. In the \textit{Folder} section of the file make sure the paths to the \textit{Spatial} and \textit{Climate} folders of the case study are correct. In these files we will be storing spatial and climate information. Also make sure the folder where Ech2o will write the results (\textit{Results} folder) exists and the path is correct.

The maps to be read by \echo will be in the PCRaster cross-system format so make sure \textsf{MapTypes = csf}. Also we will be using tables to initialize the vegetation state variables so make sure \textsf{Species\_State\_Variable\_Input\_Method = tables} . 

Turn on the reinfiltration and channel switches (1). We will use the aerodynamic resistance option as used in the Penman-Monteith equation (option 0) and the soil resistance as used by Ivanov (option 1).

We will simulate 1 year using daily time steps. We have to provide that information in seconds. The simulation will start at second 0 and will end at second 31536000. The simulation time step will be 86400. The Climate input time step and the report interval will be the same as the simulation time step (86400).

The next few sections is where the maps in the database are associated with parameters in the model. Give the appropriate file name for each parameter. A description of each parameter can be found in appendix a. At this point we should have generated all the needed files. 

The report map section is a series of boolean switches (0-1) that turn on or off the reporting (writing to the results folder) of maps with the state variables. Turn on (=1) the variables that you like reported. Mind that writing maps to the disk is an expensive processes in terms of computer time and space. 


\section{{Populating the database}}

\subsection{Creating a base map and importing the elevation model}

First we will create the spatial section of the database. Open the command line and navigate to the \textit{Spatial} folder of the case study. The maps created in this section should be placed in this folder. 

The first recommended step in the preparation of the database to run \echo is to prepare a base map holding information on the geometry of the domain grid (dimension, resolution, etc). This map can be generated when importing the digital elevation model (DEM) basemap as explained below.

The easiest way to generate the base maps is to obtain a DEM in ArcInfo ASCII raster format. \echo needs maps in planar coordinates, with lat-long coordinates in meters, such as the UTM projection. If the map is obtained in other projection using degrees a reprojection of the map is necessary using ArcGIS or any other external tool. 

Move to the example folder provided with the \echo package, open the file named \textsf{dem.asc} with a text editor and check the metadata header with information on the geometry of the raster image.

Within the PCRaster environment, type 

\begin{verbatim}
mapattr base.map
\end{verbatim}

to start the interface and crate a new blank base map named \textsf{base.map}. Introduce the number of rows and columns as indicated in the metadata of the ascii raster image. Choose the \textit{'scalar'} datatype and the \textit{'small real'} cell representation. If the projection is UTM you may want to indicate a \textit{'y increases from bottom to top'} projection. Provide the coordinates for the x upper left corner and for the y upper left corner and the cell resolution.
 
 \medskip 
\begin{Frame}
Please, note that the ArcInfo standard provides information for the lower left corner. You can calculate the value of the upper left y coordinate by adding to the lower left coordinate the result of multiplying the number of rows by the resolution.
\end{Frame}
 \medskip
 
Once this information is provided, press 'q' and answer 'y' to write the newly created map to the drive. Display the map to check it has the correct dimensions:

\begin{verbatim}
aguila base.map
\end{verbatim}
 
This base map will be used to import all other maps and to ensure all the maps in the database have the exact same geometry. To import the ArcInfo DEM map into the CSF PCRaster format type

\begin{verbatim}
asc2map -a --clone base.map dem.asc DEM.map
\end{verbatim}

This command indicates 1) that we are importing an ascii file named \textsf{dem.asc} into the PCRaster format with name \textsf{DEM.map}, 2) that the imported file has Arcinfo ascii grid format, and 3) that we are cloning the geometry of our base.map. 

Display the map to check it has been correctly imported

\begin{verbatim}
aguila DEM.map
\end{verbatim}  
  
To display it in 3D you can type 

\begin{verbatim}
aguila -3 DEM.map
\end{verbatim}

These maps will form the core of the database from which many of the other necessary maps can be derived. 

\subsection{Delineating the drainage network}

The drainage network is derived from the DEM using a steepest-descent algorithm on the 8 neighbor window around each cell. From a PCRaster environment type the command

\begin{verbatim}
pcrcalc ldd.map = lddcreate(DEM.map, 1e9,1e9,1e9,1e9)
\end{verbatim}

 \medskip 
\begin{Frame}
NOTE TO LINUX USERS:

Please, note that if you are following this tutorial in a linux computer you need to place the arguments to \textit{pcrcalc} between quotes like:

\texttt{pcrcalc 'ldd.map = lddcreate(DEM.map, 1e9,1e9,1e9,1e9)'}
 
\end{Frame}
 \medskip

This command instructs PCRaster to calculate the local drainage direction (ldd) for each cell using the dem (\textsf{DEM.map}) and save the drainage network on a map called \textsf{ldd.map}. The large numbers included as the final four arguments to the \emph{lddcreate} function are options to remove pits and core areas (see PCRaster documentation on lddcreate for more details). Display the results with aguila to visually inspect the drainage network. You may have to zoom in to see the details of the network. 

Pits and outlets are coded with the value 5 in the resulting map. These cells flow nowhere and are considered flow sinks. There is at least one sink in each basin (the outlet). Mostly we will want to have a continuous flow network towards the outlet (unless we are working on a karst area or similar), so if we see internal flow sinks it may be due to errors in the DEM that to some extent can be corrected with some of the functions in PCRaster (see PCRaster documentation for this).  

A map of the channels and the width of the channel is provided in the folder (\textsf{chanwidth.map}). Inspect it using aguila and observe that cells with a channel have a positive number indicating the width of the channel in meters and cells without a channel have attribute 0 or nodata.

The resistance presented by the channel to flow is given by Manning's $n$ coefficient. Values for Manning's $n$ coefficient needs ot be provided for each cell where the channel width is larger than 0. A map of Manning's $n$ values in $sm^{-\frac{1}{3}}$ for the example channel network is provided (\textsf{chanmanningn.map})

The parameter controlling the seepage from the subsurface system to the channel lets us fine-tune subsurface-channel interactions. A good starting value for this parameter is 0.02 for the entire channel system. The larger the value, the more resistance to flow into the channel. We can produce this map using 

\begin{verbatim}
pcrcalc chanparam.map = chanwidth.map/chanwidth.map * 0.02;
\end{verbatim}

\subsection{Defining soil and surface properties}

In this section we will create a set of maps that provide information on the soil and surface properties. Some of these properties can be derived from the DEM and for others we will use some simplifying assumptions about the spatial distribution of the properties.

The slope of the terrain can be obtained directly from the DEM using the following command 

\begin{verbatim}
pcrcalc slope.map = slope(DEM.map)
\end{verbatim}

This command will create a map named \textsf{slope.map} with the slope (rise over run) of the basin.

Now we will create a map with constant value 1 that will help us create maps of soil properties with a spatially uniform distribution.  

\begin{verbatim}
pcrcalc unit.map = DEM.map/DEM.map
\end{verbatim}

This operation divides the DEM map by itself to produce a map called \textsf{unit.map} holding 1 everywhere in the basin. 

Now we use \textsf{unit.map} to construct maps of spatially uniform properties

\begin{verbatim}
pcrcalc albedo.map = unit.map * 0.3
pcrcalc emissivity.map = unit.map * 0.98
pcrcalc soilheatcap.map = unit.map * 2.205e6
pcrcalc soilthermalK.map = unit.map * 0.2
pcrcalc dampdepth.map = unit.map * 2
pcrcalc temp_damp.map = unit.map * 10
pcrcalc snowmeltCoeff.map = unit.map * 4.1e-8
pcrcalc randrough.map = unit.map * 0.05
pcrcalc psi_ae.map = unit.map * 0.2
pcrcalc BClambda.map = unit.map *  5.3
pcrcalc KvKh.map = unit.map * 0.4
pcrcalc theta_r.map = unit.map * 0.05
pcrcalc Wc.map = unit.map * 0.7
pcrcalc Wp.map = unit.map * 9
\end{verbatim}

This will create maps of uniform albedo, surface emissivity, soil heat capacity, soil thermal conductivity, soil depth at which heat exchanges are negligible, initial soil temperature, snowmelt coefficient, terrain rugosity, soil air entry pressure, Brooks and Corey $\lambda$ parameter, vertical to horizontal hydraulic conductivity anisotropy ratio, residual soil moisture, and two soil parameter, Wc, Wp, respectively, with values equal to the multiplying scalar in the right side of the expression. 

To introduce some spatial variability in the simulation, we will assume that some geomorphologic sorting of soil particles distributes some key hydrologic properties throughout the basin. For instance, finer particles may get washed out of steep upslope areas and be deposited when water slows down in flatter areas downslope. This may produce deeper, more porous soils at the valley bottom with lower hydraulic conductivity than soils located higher up in the hillslopes. 

To simulate such a geomorphologic driven distribution we can use a topographic index that is larger for flatter cells with large contributing areas (such as valley bottoms) and smaller for steep cells near the water divide. 

\begin{verbatim}
pcrcalc topind.map = ln(accuflux(ldd.map,10000)/slope.map)
\end{verbatim}

This expression uses the function \textit{accuflux} to accumulate the area of the cells (10,000 $m^{2}$) following the drainage direction and divides it by the map of slopes (\textsf{slope.map}) that we created earlier. The function \textit{ln} takes the logarithm of the result of the quotient to equalize the distribution of values, which is highly skewed due to the exponential distribution of the accumulated areas.

We will assume that this map describes the spatial distribution of soil depth, porosity and effective hydraulic conductivity. With the help of some scaling functions we produce the resulting fields for these soil properties:

\begin{verbatim}
pcrcalc soildepth.map = topind.map 
	/areaaverage(topind.map,nominal(unit.map))
pcrcalc Keff.map = 1 / (soildepth.map * 36000)
pcrcalc poros.map = 1 / (1 + exp(0.01 * topind.map))
\end{verbatim}

We will set initial conditions for the soil assuming the basin starts free of snow, with 50\% of the pores saturated with water and with a temperature of 10$^{\circ}C$ throughout the basin:

\begin{verbatim}
pcrcalc swe.map = unit.map * 0 
pcrcalc Soil_moisture_1.map = poros.map * 0.5
pcrcalc Soil_moisture_2.map = poros.map * 0.5
pcrcalc Soil_moisture_3.map = poros.map * 0.5
pcrcalc soiltemp.map = unit.map * 10
pcrcalc streamflow.map = unit.map * 0
\end{verbatim}


We will also assume that the first hydraulic layer of the soil is 10 cm deep (0.1 m). We will also assume that the second hydraulic layer is 10 cm deep. \echo will calculate the depth of the 3rd layer such that the sum of the three layers equals the soil depth at the pixel. Additionally, the first soil layer will contain 10\% of the roots and the second layer will contain the remaining 90\%. For simplicity we further assume that the bedrock at depth of the soil is impervious (leakance=0). This parameters varies between 0 (no flow boundary) and 1 (free drainage). 

\begin{verbatim}
pcrcalc depth_soil1.map = unit.map * 0.1
pcrcalc depth_soil2.map = unit.map * 0.1
pcrcalc rootfrac1.map = unit.map * 0.1
pcrcalc rootfrac2.map = unit.map * 0.9
pcrcalc leakance.map = unit.map * 0.0
\end{verbatim}

We will see later we will spin-up the model to equilibrate the initial conditions for the characteristics and climate of the basin.

\subsection{Defining vegetation parameters}

For the sake of simplicity we will assume that there is only one type of forest homogeneously covering 60\% of the basin (proportion of area covered in each forest patch is specified in file \textit{SpecsProp.tab}). 

The parameters that define the vegetation in the forest is provided in table \ref{tab:exspecpars}

\begin{longtable}{|p{6.5cm}| l|}			
\caption{Example parameters for 1 vegetation type}			
\label{tab:exspecpars}\\			
\hline			
Parameter	&	Species 1	\\
\hline			
SpeciesID	&	1	\\
NPP/GPPRatio	&	0.47	\\
gsmax(ms-1)	&	0.09	\\
CanopyQuantumEffic(gCJ-1)	&	0.0000018	\\
MaxForestAge(yrs)	&	290	\\
OptimalTemp(C)	&	18	\\
MaxTemp(C)	&	30	\\
MinTemp(C)	&	0	\\
FoliageAllocCoef\_a	&	2.235	\\
FoliageAllocCoef\_b	&	0.006	\\
StemAllocCoef\_a	&	3.3	\\
StemAllocCoef\_b	&	6.00E-07	\\
gs\_light\_coeff	&	300	\\
gs\_vpd\_coeff	&	0.0019	\\
lwp\_d	&	0.2	\\
lwp\_c 	&	2	\\
WiltingPnt	&	0.05	\\
SpecificLeafArea(m2g-1)	&	0.003	\\
SpecificRootArea(m2kg-1)	&	0.022	\\
Crown2StemDRat	&	0.25	\\
TreeShapeParam	&	0.4	\\
WoodDens(gCm-2)	&	220000	\\
Fhdmax	&	15	\\
Fhdmin	&	5	\\
LeafTurnoverRate(s-1)	&	8.56E-09	\\
MaxLeafTurnoverWaterStress(s-1)	&	0.000000018	\\
LeafTurnoverWaterStressParam	&	0.2	\\
MaxLeafTurnoverTempStress(s-1)	&	0.000000018	\\
LeafTurnoverTempStressParam	&	0.2	\\
ColdStressParam(degC)	&	1	\\
RootTurnoverRate(s-1)	&	5.34E-09	\\
MaxCanStorageParam(m)	&	0.0000624	\\
albedo	&	0.1	\\
emissivity	&	0.95	\\
KBeers	&	0.55	\\
CanopyWatEffic(gCm-1)	&	800	\\
is\_grass	&	0	\\
DeadGrassLeafTurnoverRate(s-1)	&	0	\\
DeadGrassLeafTurnoverTempAdjustment (degC)	&	0	
\end{longtable}			

 
The parameters are listed in the order they should appear in the vegetation configuration file. Make sure you include in the first line of the header the number of species in the file and the number of information items per species (2 43). For convenience, the information in Table \ref{tab:exspecpars} is properly formatted in a parameter file named \textsf{SpeciesParams.tab}, which is is provided in the \textsf{Spatial} folder of the case study.

The next step is providing information about the variables for vegetation. There are two ways to provide this information, through tables that provide constant variable values for each initial forest patch and through maps that provide spatially variable values.
 
The easiest way is to provide first the information using tables and spin-up the model to provide maps with the distributed variables. Then restart the model using the maps as initial forest conditions. If we are using tables we need to provide a map with the spatial distribution of the types of forest or \textit{patches}. This spatial distribution is done using integer ID numbers for each patch. In this example we will assume that only one type of forest exist covering the entire area with ID 1. We can create the patch map using the unit.map:

\begin{verbatim}
pcrcalc patches.map = unit.map 
\end{verbatim}
   
The vegetation variables needed to run the model are the proportion of canopy coverage, the stem density, the leaf area index, the age, the total basal area, the species height and the root density of each species for each path. Each of these variables is contained in an individual file with the same format. 

As mentioned earlier, we will assume that the canopy of vegetation type 1 (the only type) cover 60\% of the basin. The canopy coverage file for this example would be

\begin{verbatim}
1 2
1 0.6
\end{verbatim}

Where the first element in the first line indicate the number of patches (1), the second element is the number of covers in the patch (1 vegetation type + bare soil = 2). The second line indicates the patch ID for which this line is providing information (matching the appropriate ID in the \textsf{patches.map} file). the following numbers are the proportion of the patch covered by canopy for each vegetation type (only 1 in the case). The proportion of bare soil is calculated internally from this information. The information for each species must be entered in the same order that was provided in the table of vegetation parameters including 0.0 if there is no coverage of a specific species or vegetation type in a given patch. A file (\textsf{SpecsProp.tab}) with this information is included for convenience in the \textit{example/Spatial} folder.

The same data structure is used in the files containing information for the other mandatory vegetation variables, for which files are conveniently provided: Stem density, leaf area index, stand age, total stand basal area, effective height and root density.

 
\subsection{Climate inputs} 

Navigate to the \textit{Climate} folder of the case study. The maps generated in this section need to be placed in this folder. A climate zones map provides the information to spatially distribute the climate time series and should be created first. In this example we will partition our basin in ten climate zones following the elevation contours. The easiest way to create to do that is to reclassify the DEM in ten uniform elevation zones with unique integer IDs using a classification table (see PCRaster documentation for formats):

\begin{verbatim}
<,1430] 1
<1430,1580] 2
<1580,1730] 3
<1730,1880] 4
<1880,2030] 5
<2030,2180] 6
<2180,2330] 7
<2330,2480] 8
<2480,2630] 9
<2630,> 10
\end{verbatim}

Assuming the previous table is saved under the name \textsf{ElevZonesClass.txt} we can reclassify our climate zones map with the following command:

\begin{verbatim}
pcrcalc ClimZones.map = lookupnominal(ElevZonesClass.txt,
          ..\Spatial\DEM.map)
\end{verbatim}
 
 In the \textsf{Climate} folder there are a set of example climate files providing climate information for the 10 time zones at a daily time step during 365 days starting at the beginning of the water year (October 1st). The temperature field has been generated assuming a sinusoidal cycle of temperature. A standard environmental lapse rate has been used to distribute precipitation with elevation. Precipitation for the bottom-most zone has been generated randomly to simulate a typical semiarid climate with precipitation falling during fall and winter. A linear model has been used to simulate an increase of precipitation with altitude. The other climate variables have been generated using polynomial functions to simulate seasonality and are considered to be spatially uniform. You can import the files to a spreadsheet program like MSExcel and plot them to inspect the type of climate we are simulating. 

In order to make these files usable for Ech2o we need to import them into binary format with the \textsf{asc2c} utility provided with Ech2o. This utility takes two arguments: the name of the properly formatted ascii file with the climate information and the desired name for the binary file to be written.

The following commands will import the climate files and generate the necessary binary files having the same name as the original text files but with a \textit{.bin} extension.

\begin{verbatim}
asc2c Tavg.txt Tavg.bin
asc2c Tmin.txt Tmin.bin
asc2c Tmax.txt Tmax.bin
asc2c Precip.txt Precip.bin
asc2c Sdown.txt Sdown.bin
asc2c Ldown.txt Ldown.bin
asc2c RH.txt RH.bin
asc2c windspeed.txt windspeed.bin
\end{verbatim} 

We will introduce some random variability in the precipitation field using the isohyet map assuming no autocorrelation structure or directionality of the field. The random fluctuations are produced using a uniform distribution ranging with a range 0.5-1.5 to simulate precipitation fluctuations ranging from half to one and half times the incident local precipitation 

\begin{verbatim}
pcrcalc isohyet.map = uniform(boolean(..\Spatial\unit.map))+0.5
\end{verbatim} 


\subsection{Reporting results}
 
The report time series section is another series of boolean switches that turn on or off the reporting (writing to the results folder) of time series for the specified variables. The spatial pixels for which the time series is produced are these indicated by the \textsf{TS\_mask} map. This map should contain the value zero or no data everywhere except for the pixels for which a time series is desired. These pixels should be marked with an integer ID that will be used to identify the time series in the resulting output file containing the time series information.   

One way to crate the \textsf{TS\_mask} map is by making use of a text file containing information on the location of the pixels to be monitored. This file should have one line per pixel to be monitored (probe). Each line contains the x and y coordinate of the pixel and the pixel identification number. The file \textsf{probes.txt} located in the \textsf{Spatial} folder of this case study is an example of such a file with three probes. 

Once this file is created you can import it to create a map using the PCRaster \textsf{col2map} tool. Navigate back to the \textsf{Spatial} and transform the information contained in \textsf{probes.txt} into a map using   

\begin{verbatim}
col2map --clone base.map probes.txt Tsmask.map
\end{verbatim} 
 

 
 
\subsection{Running the program}

FIRST, MAKE SURE THE FOLDER WHERE ECH2O IS TOLD TO WRITE THE RESULTS EXISTS. ECH2O WILL NOT CREATE THE FOLDER IF IT DOES NOT EXIST AND WILL TERMINATE THE RUN WHEN IT ATTEMPTS TO WRITE TO THE NON-EXISTING FOLDER.

Open the configuration file in a text editor and replace the default input file names for the soil moisture keys with the correct filenames

\begin{verbatim}
Soil_moisture_1 = Soil_moisture_1.map 
Soil_moisture_2 = Soil_moisture_2.map 
Soil_moisture_3 = Soil_moisture_3.map 
\end{verbatim} 
 
Once the database is complete and the configuration file correctly set we are ready to run Ech2o. This is simply done by navigating to the folder containing the ech2o configuration file and running the following command: 

\begin{verbatim}
ech2o config.iniv
\end{verbatim} 

Where \textsf{config.ini} stands for the name of the configuration file, typically \textsf{config.ini}, but this file can be named in any other way to differentiate different projects or runs. 

After hitting enter you will see the splash screen with the version number and a report on the pre-processing steps (whether it was able to successfully read the files and create the components of the model run). 

The screen reports information on the water mass (in $ m^{3} $ for the different components of the basin for each time step and information on the mass balance error (in \% of the total input). The mass balance error should be a very small number (typically $<$ 1.0e-10\%). If the number is large or steadily increases as the simulation progresses it is an indication of some problem in the inputs. 

Once the model has finished running you can inspect the results using \textsf{aguila} or \textsf{timeplot} to display the timeseries files or the maps in the results folder. 

For instance if you have reported discharge you can display it by typing

\begin{verbatim}
aguila OutletDisch.tab
\end{verbatim}

Spatial time series can also be displayed. For instance if you have reported snow water equivalent maps series for one year at daily timesteps (365 time steps) you can inspect them with the command

\begin{verbatim}
aguila SWE00000.001+365
\end{verbatim}

You can even drape them to the DEM. Assuming you are in the \textsf{Results} folder:

\begin{verbatim}
aguila -3 ..\spatial\DEM.map + SWE00000.001+365
\end{verbatim}

Also a file called \textsf{BasinSummary.txt} is created in the root folder (where file \textsf{config.ini} is located). This file contains summary information on the water balance of the basin in total volumes of water ($m^{3}$).
 

\subsection{Spinning-up the model}

As you see, the model diverges from the initial conditions provided and will finish with a very different spatial distribution of the state variables. Some of the variables will show a declining trend, others will show an increasing trend rather than a cycle. This indicates that the state of the model is not in equilibrium with the provided boundary conditions.

The process of running the model in order to allow it time to achieve a state of equilibrium is called 'spin-up'. The easiest way is to run the model long enough or multiple times in a loop with the same forcing until the different state variables show no significant change with respect to the previous run.


Probably the simplest way is to create a batch file that will run the model multiple times using the state-variables from the previous run as initial conditions for the following run. The first step is configure an initial run as explained in the previous example, using tables to initialize the vegetation parameters (\textsf{Species\_State \_Variable\_Input\_Method = tables}) and make sure the required state variables needed to initialize the model will be reported:
 
\begin{itemize}

\item Report\_Leaf\_Area\_Index = 1
\item Report\_Veget\_frac = 1
\item Report\_Stem\_Density = 1
\item Report\_Stand\_Age = 1
\item Report\_Root\_Mass = 1
\item Report\_Tree\_Height = 1
\item Report\_Basal\_Area = 1
\item Report\_SWE = 1
\item Report\_Soil\_Water\_Content = 1 
\item Report\_Soil\_Temperature = 1

\end{itemize}

The next step is tell the model that next time the model starts, the vegetation parameters will not be read from a table but that the parameters will be given as maps. For this set \textsf{Species\_State \_Variable\_Input\_Method = maps}.

When this variables is set to \textsf{maps}, the model expects to find a set of maps in the spatial information folder with the following names. There has to be one map per species:

\begin{itemize}

\item lai[\textit{n}].map
\item p[\textit{n}].map
\item ntr[\textit{n}].map
\item age[\textit{n}].map
\item root[\textit{n}].map
\item hgt[\textit{n}].map
\item bas[\textit{n}].map

\end{itemize}

To initiate leaf area index, species proportion in each cell, tree density, age of stands, root mass, tree height and basal area, respectively. The value for \textsf{\textit{n}} inside the square brackets is the species id \textsf{\textit{0,1...n-1}}, where \textsf{\textit{n}} is the number of species being simulated.     

To run the model in a loop we create a batch file that runs the model, takes the final state of the basin (as per the reported state variables), copies them with the right name in the spatial folder for initialization and runs the model again. Assuming you have set up the model as in the example of the next section, create a batch file with the name \textsf{spinup.bat} in the same folder where \textsf{config.ini} is located. Type the following contents into the \textsf{spinup.bat} file: 

\begin{verbatim}
@echo off
set count = 1

:loop

echo Running iteration %COUNT%

start /w ech2o config.ini

ping -w 1000 1.1.1.1 

echo finishing and copying files after iteration %COUNT%

copy /Y .\Results\root[0]0.365 .\Spatial\root[0].map
copy /Y .\Results\p[0]0000.365 .\Spatial\p[0].map
copy /Y .\Results\ntr[0]00.365 .\Spatial\ntr[0].map
copy /Y .\Results\lai[0]00.365 .\Spatial\lai[0].map
copy /Y .\Results\hgt[0]00.365 .\Spatial\hgt[0].map
copy /Y .\Results\bas[0]00.365 .\Spatial\bas[0].map
copy /Y .\Results\age[0]00.365 .\Spatial\age[0].map

copy /Y .\Results\SWE00000.365 .\Spatial\SWE.map
copy /Y .\Results\SWC1_000.365 .\Spatial\Soil_moisture_1.map
copy /Y .\Results\SWC2_000.365 .\Spatial\Soil_moisture_2.map
copy /Y .\Results\SWC3_000.365 .\Spatial\Soil_moisture_3.map
copy /Y .\Results\Ts000000.365 .\Spatial\soiltemp.map
copy /Y .\Results\Q0000000.365 .\Spatial\streamflow.map

type .\Results\lai[0].tab >> .\Results\laiacum.txt
type .\Results\NPP[0].tab >> .\Results\NPPacum.txt
type .\Results\SoilMoistureAv.tab >> .\Results\SWCacum.txt


set /A COUNT=%COUNT%+1

goto loop 

\end{verbatim}

Run the batch file by typing \textsf{spinup.bat}. This file will spinup the model until you stop it pressing \textsf{CTRL+C}. Let the model spin for a period of 5 or 10 years.

If you are reporting time series of leaf area index, net primary production and soil moisture, the batch file will append the results in a file that contains the time series for the entire spinup period. Plotting this file in Excel will let us evaluate if the state variables are equilibrated at the end of the spinup period. 







  




